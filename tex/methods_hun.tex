\section{Módszerek}
A disszertációmban bemutatott mérések nagyrészt nagy impedanciájú potenciál különbség mérések.
Az impedanciaillesztést vagy a műszer biztosította, vagy egy külső, TL082 műveleti erősítő alapú feszültségkövető áramkör.
Erre a terhelési hiba kiküszöbölése miatt volt szükség, mely a nagyon eltérő impedanciájú egységek illesztésekor lép fel.
Ilyen hatás léphet fel például mikroelektródok potenciáljának mérése során.

Szinte az összes méréshez házi készítési PEKM egységet és saját készításű szoftvert használtam.
Az összes bemutatott ion szelektív elektród saját készítésű volt.
A következő indikátor elektródokat használtam:

\begin{itemize}
\item pH érzékeny antimon mikroelektród.
\item pH érzékeny volfrám mikroelektród.
\item Magnézium-ion szelektív mikropipetta.
\item Kálium-ion szelektív mikropipetta.
\end{itemize}

Az indikátor elektród potenciálját mindig egy Ag/AgCl/(3 M) referencia félcellával szemben mértem.
A dekonvolúciót, a diffúzió és PEKM szimulációkat FORTRAN nyelven írtam.
