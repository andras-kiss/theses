\section{Konklúzió és tézispontok}
%\addcontentsline{toc}{chapter}{Conclusions}
%\pagestyle{plain}
A disszertációmban bemutatott munkámmal a célom a potenciometriás pásztázó elektrokémiai mikroszkópia javítása volt.
A technika viszonylag lassú, a potenciometriás mérőcellák nagy válaszideje miatt.
Sok esetben előnyös lenne a gyorsabb pásztázás, például gyorsan változó rendszerek esetén.
Az ehhez szükséges nagy sebesség alkalmazásának következtében azonban a kép torzított lesz.
Munkámmal sikeresen felgyorsítottam a módszert, a kép minőségének romlása nélkül.
Ezen kívül sikeresen megmutattam, hogy bizonyos mérésekben tapasztalt ellentmondások magyarázata az elektromos mező közvetlen hatása a mért potenciálra.

Főbb eredményeimet a \textbf{tézis pontokban} összegzem:

\begin{enumerate}
\item Sikeresen rövidítettem a poteciometriás cella válaszidejét szilárd-kontaktusú mikroelektródok alkalmazásával, így az ellenállás csökkentésével.
Bemutattam, hogy ezzel milyen mértékben javítható a potenciometriás PEKM képalkotása.
Több paraméterét összehasonlítottam a konvencionális folyadék-kontaktusú mikroelektródokéval.
A javulást modellrendszer vizsgálatával szemléltettem.

\item Felhasználva az új szilárd-kontaktusú elektródok előnyös tulajdonságait, magnézium és magnézium-ötvözet galvanikus korrózióját vizsgáltam a kioldódott ionok koncentráiójának minta feletti térképezésével.
Az új típusú mérőcsúcsok gyorsabb pásztázást tettek lehetővé.

\item PEKM mérések alapján megbecsültem a korróziós áramot AZ63 és vas minták között, és összehasonlítottam a korróziós áram indirekt mérésével.
A két módszerrel nagyon hasonló eredményt kaptam, jelezve az új mikroelektródok alkalmazhatóságát hasonló, kvantitatív mérésekben.

\item Kidolgoztam két, körszimmetrikus rendszerekre optimalizált pásztázási mintázatot, és hozzájuk tartozó algoritmusokat.
Numerikus szimulációkkal megmutattam, hogy ezek használatával a körszimmetrikus rendszerekről alkotott képek jóval kisebb torzításúak.
A szimulációkat kísérletes eredményekkel igazoltam.

\item Elsőként használtam dekonvolúciót potenciometriás PEKM képek torzításának csökkentésére.
Megmutattam, hogy a nagy időállandó okozta torzítás jelentősen csökkenthető.
A módszert jóságának igazolására a feldolgozott képeket összehasonlítottam sokkal lassabban rögzített képekkel, melyek gyakorlatilag torzítás mentesek voltak.

\item Dekonvolúcióval helyreállítottam szénacél mintáról készült PEKM képeket.
A gyors pásztázás miatti torzítás jelentősen csökkent a feldolgozás után.
Az eredmény kiértékelését a nagy időállandó okozta torzulás eredetileg ellehetetlenítette.

\item Bemutattam a ,,vak dekonvolúció'' lehetőségét is.
Ezen módszer használata olyan esetekben szükséges, ahol nem ismert a dekonvolúciós függvény minden paramétere.

\item Bebizonyítottam, hogy a galvanikus korrózió során fellépő elektromos mező jelentős hibát okoz a kiértékelés során.
Az elektromos mező lokális értéke hozzáadódik a mért potenciálhoz.
Az általam vizsgált esetben a hiba négy nagyságrendes tévedést okozott volna.
A hatás figyelembe vételével a hiba kiküszöbölhető.
\end{enumerate}
