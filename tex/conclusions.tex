\section{Conclusions}
%\addcontentsline{toc}{chapter}{Conclusions}
%\pagestyle{plain}
The present work has been devoted to improve potentiometric Scanning Electrochemical Microscopy.
Scanning is relatively slow due to the long response time of the potentiometric measuring cell.
Shortened scanning time is useful when the studied system is changing.
When scanned too fast however, distortion is added to the image.
I've successfully sped up the technique without compromising image quality.
In another effort, I've managed to separate the effect of electric field from the Nernstian potential response of the ion selective microelectrode. 

The main results are summarized in the \textbf{thesis points}:

\begin{enumerate}
\item I've successfully shortened response time of the potentiometric cell by using low resistance, solid-contact microelectrodes.
I've compared them to conventional, liquid contact microelectrodes by basic characterization and model system study to prove the improved performance.

\item Taking adventage of the new solid-contact electrodes, I've studied the galvanic corrosion of magnesium and the AZ63 magnesium alloy by mapping the concentration of dissolving ions.
I used the new solid contact ion selective microelectrodes as SECM probes. This allowed faster scan rates.

\item I've estimated the corrosion current based on the SECM measurements, and compared the result with that obtained with another, established method; the indirect measurement of corrosion current.
After applying Faraday's Law of Electrolysis, the two results could be compared.
They were very similar, suggesting the applicability of SECM in obtaining quantitative results.

\item I've designed new scanning patterns and algorithms, optimized to radially symmetric targets.
I've proven that with these new patterns and algorithms, image distortion is lower compared to the conventional ones, by numerical simulations and experimental SECM scans.

\item I've shown that by using deconvolution, \emph{RC} distortion can be significantly lowered in the potentiometric SECM images.
To prove the validity of the technique, I've compared deconvoluted images to equilibrium images scanned at a rate which allowed to record equilibrium potentials.

\item I've used deconvolution to restore potentiometric SECM images about a corroding carbon steel sample.
Evaluation of this data was possible, because scanning time \emph{and} distortion was reduced at the same time.

\item I've shown the applicability of blind deconvolution.
This method can be used on measurements where the convolution function cannot be determined.

\item I've proven that the electric field present in many studied systems -- galvanically corroding ones in particular -- affects the measured potential.
The electric field has a direct influence on the measured potential, which is then a sum of this contribution and the Nernstian response associated with ion activity.
This effect can cause serious errors in interpretations in the measurements.
In this case, the error was almost four orders of magnitude.
By taking this effect into account, a more accurate conclusion can be drawn.

\end{enumerate}
