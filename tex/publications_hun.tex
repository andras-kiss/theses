%\begin{center}
%\emph{Bírált folyóiratokban megjelent közlemények száma:} 12
%\emph{Kumulatív IF:} 35.063
%\emph{Hivatkozások:} 88
%\emph{h-index:} 5
%\end{center}
\section{Közlemények}

\subsection{A disszertációhoz kapcsolódó bírált közlemények}
\begin{enumerate}

\item Ricardo M. Souto, \textbf{András Kiss}, Javier Izquierdo, Lívia Nagy, István Bitter, Géza Nagy, Spatially-resolved imaging of concentration distributions on corroding mag\-ne\-si\-um-based materials exposed to aqueous environments by SECM, \emph{Electrochemistry Communications 26 (2013): 25-28.}, IF.: 4.85, cited by: 31

\item \textbf{András Kiss}, Ricardo M. Souto, Géza Nagy, Investigation of Mg/Al alloy sacrificial anode corrosion with Scanning Electrochemical Microscopy, \emph{Periodica Polytechnica Chemical Engineering 57, no. 1-2 (2013): 11-14.}, IF.: 0.30, cited by: 5

\item Javier Izquierdo, \textbf{András Kiss}, Juan José Santana, Lívia Nagy, István Bitter, Hugh S. Isaacs, Géza Nagy, Ricardo M. Souto, Development of Mg$^{2+}$ ion-selective microelectrodes for potentiometric scanning electrochemical microscopy monitoring of galvanic corrosion processes, \emph{Journal of The Electrochemical Society 160, no. 9 (2013): C451-C459.}, IF.: 3.27, cited by: 23

\item \textbf{András Kiss}, Géza Nagy, New SECM scanning algorithms for improved potentiometric imaging of circularly symmetric targets, \emph{Electrochimica Acta 119 (2014): 169-174.}, IF.: 4.50, cited by: 8

\item \textbf{András Kiss}, Géza Nagy, Deconvolution of potentiometric SECM images recorded with high scan rate, \emph{Electrochimica Acta 163 (2015): 303-309.}, IF.: 4.50, cited by: 7

\item \textbf{András Kiss}, Géza Nagy, Deconvolution in potentiometric SECM, \emph{Electroanalysis 27, no. 3 (2015): 587-590.}, IF.: 2.14, cited by: 2


\item \textbf{András Kiss}, Dániel Filotás, Ricardo M Souto, Géza Nagy, The effect of electric field on potentiometric Scanning Electrochemical Microscopic imaging, \emph{Electrochemistry Communications 77 (2017): 138-141.}, IF.: 4.569
\end{enumerate}

\subsection{A disszertációhoz kapcsolódó előadások és poszterek}
\begin{enumerate}
\item Investigation of Mg/Al alloy sacrificial anode corrosion with Scanning Electrochemical Microscopy, Poster, \emph{Chemical Engineering Workshop ’12, Veszprém, 2012.}

\item Investigation of galvanic corrosion of the Fe-Mg galvanic pair with Scanning Electrochemical Microscope, Poster, \emph{Chemical Sensors Workshop ’12, Pécs, 2012.}

\item Fabrication of a new, solid contact Mg$^{2+}$ ion-selective electrode, and its application in Scanning Electrochemical Microscopic corrosion studies, Presenttion, \emph{1st Doctoral Workshop on Natural Sciences, Pécs, 2012.}

\item A new, solid contact Mg$^{2+}$ ion-selective electrode as measuring tip for Scanning Electrochemical Microscope in corrosion studies, Presentation, \emph{János Szentágothai Memorial Conference and Student Competition, Pécs, 2012 October 29-30.}

\item New insights in the corrosion mechanism of magnesium by SECM, Presentation, \emph{7th Workshop on Scanning Electrochemical Microscopy (SECM) and Related Techniques, Ein Gedi, Israel, February 17-21, 2013.}

\item High-speed potentiometric SECM imaging of radially symmetric targets, Presentation, \emph{ESEAC Malmö, Sweden, 11-14 June 2013.}

\item Deconvolution of potentiometric SECM images recorded with high scanrate, Poster, \emph{Mátrafüred Conference 2014 Június 13-16, Visegrád, Hungary.}

\item High-speed SECM imaging, Plenar presentation, \emph{Analytica Conference 2016 May 10-13, München, Germany.}
\end{enumerate}

\subsection{A disszertációhoz nem kapcsolódó bírált közlemények}
\begin{enumerate}
\item \textbf{András Kiss}, László Kiss, Barna Kovács, Géza Nagy, Air Gap Microcell for Scanning Electrochemical Microscopic Imaging of Carbon Dioxide Output. Model Calculation and Gas Phase SECM Measurements for Estimation of Carbon Dioxide Producing Activity of Microbial Sources, \emph{Electroanalysis 23, no. 10 (2011): 2320-2326.}, IF.: 2.14, cited by: 3

\item Ricardo M. Souto, Javier Izquierdo, Juan José Santana, \textbf{András Kiss}, Lívia Nagy, Géza Nagy. Progress in scanning electrochemical microscopy by coupling potentiometric and amperometric measurement modes, \emph{Current Microscopy Contributions to Advances in Science and Technology, Formatex Research Center, Badajoz (2012): 1407-1415}, cited by: 3

\item Lívia Nagy, Gergely Gyetvai, \textbf{András Kiss}, Ricardo Souto, Javier Izquierdo, Géza Nagy, Speciális célra szolgáló mikroelektródok kifejlesztése és alkalmazása, \emph{Magyar Kémiai Folyóirat 119, 2-3. (2013): 104-109.}

\item Zsuzsanna \H{O}ri, \textbf{András Kiss}, Anton Alexandru Ciucu, Constantin Mihailciuc, Cristian Dragos Stefanescu, Lívia Nagy, Géza Nagy, Sensitivity enhancement of a ,,bananatrode'' biosensor for dopamine based on SECM studies inside its reaction layer, \emph{Sensors and Actuators B: Chemical 190 (2014): 149-156.}, IF.: 4.10, cited by: 4

\item Javier Izquierdo, Bibiana M Fernández-Pérez, Dániel Filotás, Zsuzsanna Őri, \textbf{András Kiss}, Romen T Martín-Gómez, Lívia Nagy, Géza Nagy, Ricardo M Souto, Imaging of Concentration Distributions and Hydrogen Evolution on Corroding Magnesium Exposed to Aqueous Environments Using Scanning Electrochemical Microscopy, \emph{Electroanalysis 28, (2016): 2354-2366.}, IF.: 2.471, cited by: 2

\item A. El Jaouhari,  Dániel Filotás, \textbf{András Kiss}, M. Laabd, E. A. Bazzaoui, Lívia Nagy, Géza Nagy, A. Albourine, J. I. Martins, R. Wang, SECM investigation of electrochemically synthesized polypyrrole from aqueous medium, \emph{Journal of Applied Electrochemistry 46 (2016): 1199-1209.}, IF.: 2.223

\end{enumerate}

\subsection{A disszertációhoz nem kapcsolódó előadások és poszterek}
\begin{enumerate}
\item CO$_2$ Partial Pressure Imaging in Gas Phase with Scanning Electrochemical Microscopy (SECM), Poster, \emph{X. CECE Conference, Pécs, 2010.}

\item Selective Amperometric Determination Of Pyrocatechol and Phenol in Wines with Flow-Injection Analysis, Poster, \emph{X. CECE Conference, Pécs, 2010.}

\item Four-Channel Enzyme Biosensor for Determination of Phenols in Wine, Poster, \emph{X. CECE Conference, Pécs, 2010.}

\item Development of a CO$_2$ microcell, and its application as measuring tip in Scanning Electrochemical Microscope. Scanning in gas phase over biological samples, Presentation, \emph{XXXIV. Szegedi Kémiai Előadói Napok, Szeged, 2011.}
\end{enumerate}
