\section{Methods}
Most of the measurements in my dissertation are high impedance potential measurements. Impedance matching was either provided by the instrument or by a home made TL082 operational amplifier based voltage follower circuit. This was needed to avoid loading error present in measuring high impedance voltage sources, such as potentiometric cells employing ion selective microelectrodes.

The SECM instrument and the software controlling its movement was home made. All the ion selective microelectrodes were home made. I have used several different kind of microelectrodes:

\begin{itemize}
\item Antimony pH sensitive microelectrodes.
\item Tungsten pH sensitive microelectrodes.
\item Magnesium ion selective micropipettes.
\item Potassium ion selective micropipettes.
\end{itemize}

Potential was always measured against an Ag/AgCl/(3 M) reference half cell. The deconvolutions and the diffusion/SECM simulations were performed by a program written in FORTRAN.
