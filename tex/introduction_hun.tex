\section{Bevezetés}
A Pásztázó Alagúthatás Mikroszkópia (STM, Scanning Tunnelling Microscopy) 1981-es feltalálása óta a felületanalízis hatalmas fejlődésnek indult.
A módszer fontosságát jelzi, hogy megalkotójuk, Binni és Rohrer mindössze öt évvel később, 1986-ban Bonel-díjban részesültek úttörő munkájukért. 
Ez a technika volt az első Pásztázó Mérőcsúcs Mikroszkópiás (SPM, Scanning Probe Microscopy) technika, melyet sok hasonló követett.
Közös bennük az a helyi mérés, melyet egy raszter minden egyes előre meghatározott pontján végzünk el.
A mért paraméter helykoordináták mentén való ábrázolásával rajzolható meg a mikroszkópiás kép.
Legfőbb előnyük a hagyományos optikai mikroszkópiához képest az elképesztő felbontásuk.
Mivel a technikát nem korlátozza az Abbes képlet, egyedi atomok is ,,láthatóvá'' tehetőek.
Az eredeti STM technika változatai sorra jelentek meg az elkövetkező években.
Az Atomerő Mikroszkópiát például 1982-ben találta fel, ugyancsak Binnig és Rohrer.

1989-ben alkották meg elektrokémikusok a módszer elektrokémiai változatát, a Pásztázó Elektrokémiai Mikroszkópiát (PEKM, SECM = Scanning Electrochemical Microscopy).
A működési elv ugyanaz, a különbség csak a mérőcsúcsban rejlik.
Ebben az esetben ez egy mikroelektród.
A technikával kémiai információ térképezhető nagy felbontással, felületek egész soráról.
Legnagyobb hátránya a viszonylag kis sebessége, mely a pásztázási folyamat következménye.
Az egész képet ugyanazzal a mérőcsúccsal kell végigpásztázni, szemben például az optikai mikroszkópiával, ahol leggyakrabban egy szenzor mátrixot alkalmazunk képalkotási célra.
Ennek az a következménye, hogy minél több pontól áll a kép, annál tovább tart a képalkotás.
Ez különösen nagy probléma a PEKM potenciometriás üzemmódjában.
A potenciometriás cella válaszidejét az időállandó írja le, mely főleg az indikátor elektród ellenállásától függ.
A mikroelektródok kis méretükből adódóan rendkívül nagy ellenállásúak lehetnek, elérve akár a G$\ohm$-os tartományt is.
Ezért a képalkotás ideje jelentősen megnyúlhat, és általában percekben vagy akár órákban mérhető.

A többi pásztázó technika jelentősen gyorsult az utóbbi néhány évtizedben, és sebességük lehetővé teszi videók rögzítését is.
Az alacsony sebesség azonban egy sokszor elhanyagolt korlátja a PEKM technikának, ami akadályozza a gyors, és nagyfelbontású képek rögzítését.
A kép vagy gyorsan elkészül de torzított lesz, vagy jó minőségű, de nem pillanatszerű.
Így a kép egyes pontjainak nem csak eltérő térbeli, de eltérő időbeli koordinátái is lesznek.

A disszertációmban ennek a problémának a megoldására irányuló munkámat írom le, és bemutatok három megoldást, melyeket kidolgoztam:

\begin{enumerate}
\item Újszerű, szilárd kontaktusú mikroelektródok használata mérőcsúcsként a hagyományos, folyadék kontaktusúak helyett.
\item A pásztázási mintázatok és algoritmusok optimalizálása.
\item A nagy sebességgel rögzített, de torzított képek dekonvolúciója.
\end{enumerate}

Első megközelítésben megpróbáltam az indikátor elektród ellenállását csökkenteni.
A hagyományos folyadék kontaktus helyett, vezető polimer alapú, szilárd belső kontaktus használatával a potenciometriás mérőcella ellenállása, és így időállandója csökkenthető.
Vezető polimereket használtak már korábban, de nem célzottan magnézium ion-szelektív mikroelektródok ellenállásának csökkentésére.

A második módszer amit kidolgoztam a pásztázási mintázatok és algoritmusok optimalizálása.
A legtöbb tanulmányozott rendszer meghatározott, egyszerű szimmetriával bír, melyre optimalizálható a pásztázás, a kisebb torzítás érdekében.
Egy egyszerű, de gyakori szimmetriát, a körszimmetriát választottam, és kidolgoztam két optimalizált pásztázási mintázatot és hozzájuk tartozó algoritmust.

A harmadik technika amit alkalmaztam, a képfeldolgozás volt.
A potenciometriás cella időbeli válaszát leírő függvény jól ismert.
Néhány könnyen mérhető paraméter birtokában egy dekonvolúciós függvény megadható.
Ennek segítségével az egyensúlyi potenciál megjósolható a kép minden egyes mintavételi pontjára, és a torzítás drasztikusan csökkenthető.

Ezen technikák hatékonyságát először modellrendszereken vizsgáltam, majd korróziós tanulmányokban mutattam be alkalmazhatóságukat.
Több együttműködés során használtam a kidolgozott módszereket, és néhány eredményt ezek közül is bemutatok a disszertációmban.

Vizsgáltam továbbá az elektromos tér hatását a potenciometriás PEKM képalkotásra, mely sok vizsgált rendszerben jelen van.
Bizonyos esetekben, ahol nagy potenciálkülönbség van a vizsgált rendszer különböző pontjai között, erős elektromos tér alakul ki.
Ilyen például a galvanikus korrózió, mely során nagy potenciálkülönbség van a galvánpárok felületei között.
Az elektromos potenciál mérőcsúcsnál jelentkező értéke befolyásolja a mért értéket.
Megvizsgáltam ezen hatás nagyságát, és megpróbáltam elkülöníteni az elektromos tér hatását.
